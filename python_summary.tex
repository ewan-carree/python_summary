\documentclass[a4paper, 12pt]{article}
\usepackage[utf8]{inputenc}
\usepackage[T1]{fontenc}    
\usepackage[english]{babel}

\title{\textbf{python\_summary}}
\author{Ewan Carrée}
\date{August 2020}

\usepackage{listings}
\usepackage{xcolor}

%New colors defined below
\definecolor{codegray}{rgb}{0.5,0.5,0.5}
\definecolor{backcolour}{rgb}{0.95,0.95,0.92}

%Code listing style named "mystyle"
\lstdefinestyle{mystyle}{
  backgroundcolor=\color{backcolour},   commentstyle=\color{codegray},
  keywordstyle=\color{magenta},
  numberstyle=\tiny\color{codegray},
  stringstyle=\color{black},
  basicstyle=\ttfamily\footnotesize,
  breakatwhitespace=false,         
  breaklines=true,                 
  captionpos=b,                    
  keepspaces=true,                 
  numbers=left,                    
  numbersep=5pt,                  
  showspaces=false,                
  showstringspaces=false,
  showtabs=false,                  
  tabsize=4
}
%"mystyle" code listing set
\lstset{style=mystyle}

%\usepackage[top=2cm, bottom=2cm, left=2cm, right=2cm]{geometry} 

\usepackage{color}
\usepackage{hyperref} %links
\hypersetup{
    colorlinks=true,
    linkcolor=black,
    urlcolor=blue,
    linktoc=all
}

\iffalse
%Python code highlighting
\begin{lstlisting}[language=Python]

\end{lstlisting}
\fi

\begin{document}

\maketitle
\newpage

\tableofcontents

\newpage
\section{Python reserved words}
In python 3, there are some reserved words that you can't use in your program as variables, function name, class name, ... \\ \\ \\

\begin{table}[h]
\begin{center}
{\renewcommand{\arraystretch}{2} %donne la distance entre les lignes%
{\setlength{\tabcolsep}{1.5cm} %donne la distance entre les collones%
\begin{tabular}{|l|c|r|}
  \hline
  \hyperref[subsec:Bool]{False} & None & \hyperref[subsec:Bool]{True} \\
  \hline
  and & \hyperref[As]{as} & \hyperref[subsec:Assertion]{assert} \\
  \hline
  \hyperref[subsec:BCPR]{break} & \hyperref[sec:Class]{class} & \hyperref[subsec:BCPR]{continue} \\
  \hline
  \hyperref[sec:Function]{def} & del & \hyperref[IEE]{elif} \\
  \hline
  \hyperref[IEE]{else} & \hyperref[TEEF]{except} & \hyperref[TEEF]{finally} \\
  \hline
  for & from & \hyperref[subsec:Global]{global} \\
  \hline
  \hyperref[IEE]{if} & \hyperref[subsec:Import]{import} & \hyperref[subsec:ListComprehension]{in} \\
  \hline
  is & \hyperref[subsec:Lambda]{lambda} & nonlocal \\
  \hline 
  not & or & \hyperref[subsec:BCPR]{pass} \\
  \hline 
  \hyperref[Raise]{raise} & \hyperref[subsec:BCPR]{return} & \hyperref[TEEF]{try} \\
  \hline
  while & \hyperref[subsec:ContextManager]{with} & \hyperref[sec:Generators]{yield} \\
  \hline
\end{tabular}}}
\end{center}
\caption{Reserved words}
\end{table}



\clearpage
\section{Types}
Python offers different basics types that are often enough for your program.\newline

\subsection{Immutable}
Immutable means that you can't directly modify your variable after you've assigned it. You have to reassign or create a new variable to work on a previous one\newline

\subsubsection{String}
Python string is an ordered collection of characters which is used to represent and store the text-based information. Strings are stored as individual characters in a contiguous memory location. It can be accessed from both directions: forward and backward. A string can be seen in some ways as a list of characters. \\It's better to use ' for as single character and " for a chain. \\ \\
Here are some methods that you can apply on strings (I choose to reassign the variable whenever it's possible) : 
\begin{itemize}
\begin{lstlisting}[language=Python]
string = "Hello, world!"
\end{lstlisting}

\item Upper case \\
Return a string with every single element from the original one in uppercase.
\begin{lstlisting}[language=Python]
string = string.upper()
# HELLO, WORLD!
\end{lstlisting}

\item Lowercase \\
Return a string with every single element from the original one in lowercase.
\begin{lstlisting}[language=Python]
string = string.lower()
# hello, world!
\end{lstlisting}

\item Captital \\
Return a string with the first Letter in uppercase and the rest in lowercase.
\begin{lstlisting}[language=Python]
string = string.capitalize()
# Hello, world!
\end{lstlisting}

\item Strip \\
Return a string whithout the blank spaces from both sides of the original string.
\begin{lstlisting}[language=Python]
string = string.strip()
\end{lstlisting}

\item Find \\
Return an integer which is the position of the first argument found in the string from left to right.
\begin{lstlisting}[language=Python]
position = string.find('l')
# 2
\end{lstlisting}

\item Count \\
Return an integer that represents the number of occurences of the argument found in the string. 
\begin{lstlisting}[language=Python]
nb = string.count('l')
# 3
\end{lstlisting}

\item Replace \\
Return a string where at least a part of it have been replaced by some another words.
\begin{lstlisting}[language=Python]
string = string.replace("world", "dlrow")
# Hello, dlrow!
\end{lstlisting}

\item Multiply \\
Return a string with n times the original string.
\begin{lstlisting}[language=Python]
string = string*3
# Hello, dlrow!Hello, dlrow!Hello, dlrow!
\end{lstlisting}

\item String to List \\
Transform your string into a list.
\begin{lstlisting}[language=Python]
tableau = string.split(' ')
# ['Hello,', 'dlrow!Hello,', 'dlrow!Hello,', 'dlrow!']
\end{lstlisting}

\item Fill \\
Return a string filled with zero to correspond to have the same digit in every number.
\begin{lstlisting}[language=Python]
string = '5'
string = string.zfill(3)
# 005
\end{lstlisting}
\end{itemize}

\subsubsection{Integer}
You can write very long number in a more readable way with \_ : 
\begin{lstlisting}[language=Python]
nb = 1_000_000
# 1000000
\end{lstlisting}

\subsubsection{Float}
A float is a decimal number, it's most precise than an integer. \\
\begin{itemize}
\item Decimal handling \\
Sometimes you don't need to conserve every decimal from a variable. You can modify it by two different ways : 
\begin{lstlisting}[language=Python]
my_float = 3.3333333
print("%.0f" % my_float + "with no decimal and " + "%.2f" % my_float + "with two decimals") #3 with no decimal and 3.33 with two decimals
round(my_float, 3) # 3.333
\end{lstlisting}
\end{itemize}

\subsubsection{Tuple}
A tuple is a container that can contain multiple variables with multiple types that have a link together. \\
\begin{itemize}
\item Ignore unpacked variables \\
When you unpack tuple arguments, sometimes some argument aren't useful, the common way to write it is with \_
\begin{lstlisting}[language=Python]
my_tuple = (1,2)
my_tuple2 = (1,2,3,4,5)

#unpack :
a, b = my_tuple #1 and 2

#We don't need b :
a, _ = my_tuple #1

#Another exemple :
for _ in range(2):
	print("I d'ont care about i")
#I d'ont care about i
#I d'ont care about i

#Another exemple :
a, b, *_ = my_tuple2 #We ignore everythin after the first two variables, the rest is stored in a list.
a, b, *c, d = my_tuple2 
print(f"{a}, {b}, {c}, {d}") #1, 2, [3, 4], 5
\end{lstlisting}
\end{itemize}


\subsubsection{Bool}
\label{subsec:Bool}
There is only two types of boolean : True and False. \\
Everything can be considered as True if it diffrent from 0 or False. \\
\begin{lstlisting}[language=Python]
a = "Salut"
b = 0
if a:
	print("a fonctionne")
	# a fonctionne
else:
	print("a ne fonctionne pas")

if b:
	print("b fonctionne")
else:
	print("b ne fonctionne pas")
	# b ne fonctionne pas
\end{lstlisting}

\subsection{Mutable}
Mutable means that if you are working on a variable, you directly modify it and don't need to create a new variable or reassign the previous one. It cans cause conflict if two process are using the same variable that can be changed.\newline

\subsubsection{List}
A list is a container that can contain multiple variables with multiple types that have a link together.\\ \\
Here are some methods that you can apply on lists : 

\begin{lstlisting}[language=Python]
tableau = ['Hello,', 'dlrow!Hello,', 'dlrow!Hello,', 'dlrow!']
\end{lstlisting}

\begin{itemize}
\item Slicing \\
Slicing is a method that concerns containers, you can use slicing on every possible container. \textbf{[start:end:step]} \textit{end argument isn't take in account : [start:end[}
\begin{itemize}
    \item Last element \\
    If you don't know the length of a certain container, you can access to the last element with slicing.
    \begin{lstlisting}[language=Python]
    dernier_elem = tableau[-1]
    # dlrow!
    \end{lstlisting}
    
    \item Trunc \\
    You can also trunc your list to keep just what you need.
    \begin{lstlisting}[language=Python]
    first_part = tableau[:3]
    second_part = tableau[3:]
    # first_part : ['Hello,', 'dlrow!Hello,', 'dlrow!Hello,'] and second_part : ['dlrow!']
    \end{lstlisting}
    
    \item Insert \\
    If you need to precisely insert an element into list, slicing is a good way of doing it.
    \begin{lstlisting}[language=Python]
    tableau[1:1] = 'A'
    # ['Hello,', 'A', 'dlrow!Hello,', 'dlrow!Hello,', 'dlrow!']
    \end{lstlisting}
\end{itemize}

\item Add \\
You can merge two lists into single one containing elements from both original lists.
\begin{lstlisting}[language=Python]
new_tab = first_part + second_part
# ['Hello,', 'dlrow!Hello,', 'dlrow!Hello,', 'dlrow!']
\end{lstlisting}

\item Multiply \\
Just like a string, you can multiply your list.
\begin{lstlisting}[language=Python]
multilpied_new_tab = new_tab *3
# ['Hello,', 'dlrow!Hello,', 'dlrow!Hello,', 'dlrow!', 'Hello,', 'dlrow!Hello,', 'dlrow!Hello,', 'dlrow!', 'Hello,', 'dlrow!Hello,', 'dlrow!Hello,', 'dlrow!']
\end{lstlisting}

\item Index \\
Return an integer which correspond to the position of the element searched.
\begin{lstlisting}[language=Python]
position = tableau.index('A')
# 1
\end{lstlisting}

\item List to String \\
Transfrom your list into a string.
\begin{lstlisting}[language=Python]
string = ' '.join(tableau)
# Hello, A dlrow!Hello, dlrow!Hello, dlrow!
\end{lstlisting}
\end{itemize}

\subsubsection{Dict}
A dict is an non-organized container that have variables associated to keys to recognize them. Keys are offently strings but we can also use integers.\\ \\
Here are some methods that you can apply on dicts : 
\begin{lstlisting}[language=Python]
dictionnaire = {"nom":"Jerome", "age":20}
\end{lstlisting}

\begin{itemize}
\item Keys \\
Return a (dict)list containing every keys that are into you dictionnary.
\begin{lstlisting}[language=Python]
keys = dictionnaire.keys()
#dict_keys(['nom', 'age'])
\end{lstlisting}

\item Values \\
Return a (dict)list containing every values that are into you dictionnary.
\begin{lstlisting}[language=Python]
keys = dictionnaire.keys()
#dict_values(['Jerome', 20])
\end{lstlisting}

\item Items \\
Return a (dict)tuple containing every keys associated to their values that are into you dictionnary.
\begin{lstlisting}[language=Python]
both = dictionnaire.items()
# dict_items([('nom', 'Jerome'), ('age', 20)])
\end{lstlisting}
\end{itemize}

\subsubsection{Set}
A set is a container that are very useful for mathematical operations\\
Here are some methods that you can apply on sets : 
\begin{itemize}
\item
\end{itemize}

\newpage
\section{Class}
\label{sec:Class}
Classes are a way of bringing together data and functionality. Creating a new class creates a new type of object and so new instances of this type can be built. Each instance can have its own attributes, which defines its state. An instance can also have methods (defined by the instance class) to modify its state.\newline
\subsection{Instance}
\subsection{Methods}
\subsection{Dunder methods}
\subsection{Static and class methods}
\subsection{Property}
\subsection{Inheritance}
\subsection{Metaclass}

\newpage
\section{Function}
\label{sec:Function}
A function is a block of organized, reusable code that is used to perform a single, related action. Functions provide better modularity for your application and a high degree of code reusing.\newline

\subsection{Lambda function}
\label{subsec:Lambda}
Lambda function are very small function, it's very simple to create them and very often better than a function. \textbf{function\_name = lambda parameter : return\_value}
\begin{lstlisting}[language=Python]
def equivalent_lambda(x):
	return x+5
	
real_lambda = lambda x : x+5

# equivalent_lambda : 10 and real_lambda : 10
\end{lstlisting}

\subsection{Default parameters}
Default parameters are used for parameters that don't necessarily need a value. We assign a default value that can be overloaded if needed. Default parameters are always declared after non-default parameters into function's head. \textbf{def func(non\_default\_parameters, \textit{default\_parameters = value})}

\begin{lstlisting}[language=Python]
def func(age, nom="Jerome"):
	print(f"Tu as {age} ans et tu t'appelles {nom}")

func(20, "Antoine")
func(20)
# Tu as 20 ans et tu t'appelles Antoine
# Tu as 20 ans et tu t'appelles Jerome

\end{lstlisting}

\subsection{Map function}
Map function is a special function used to apply a function for each element of a container. \textbf{map(function\_to\_apply, list)}
\begin{lstlisting}[language=Python]
liste = [1,2,3]
def map_function(x):
	return x**x
mapped_list = list(map(map_function,liste)) #No () for argument function

# before : [1, 2, 3] and now : [1, 4, 27]
\end{lstlisting}

\subsection{Filter function}
Filter function is a special function used to apply a selection for each element of a container. \textbf{map(function\_to\_apply, list)}
\begin{lstlisting}[language=Python]
liste = [1,2,3]
def filter_function(x):
	return x%2!=0
filtered_list = list(filter(filter_function,liste))

# before : [1, 2, 3] and now : [1, 3]
\end{lstlisting}

\subsection{List Comprehension}
List comprehension is probably one of the most powerful method that you can apply. It conbines map and filter function into a single way to do it. \textbf{[func\_to\_apply() for elem in list if filter\_condition]}
\label{subsec:ListComprehension}
\begin{lstlisting}[language=Python]
liste = [1,2,3]
new_list = [map_function(x) for x in liste if x%2!=0]

#before : [1, 2, 3] and now : [1, 27]
\end{lstlisting}

\subsection{Unknown parameters}
Sometimes you don't know how many arguments a function is going to receive. Also, it's important to handle unknown parameters. \textit{It's the case for exemple with the print() function}. \textbf{* means unnamed parameters and ** means named parameters}
\begin{lstlisting}[language=Python]
def parameters(*args, **kwargs):
	print(f"I received these unnamed args : {args}")
	print(f"I received these named args : {kwargs}")
parameters(1,"azerty",[1,2], couleur="rouge", taille_en_cm=172)
both
# I received these unnamed args : (1, 'azerty', [1, 2])
# I received these named args : {'couleur': 'rouge', 'taille_en_cm': 172}
\end{lstlisting}

\newpage
\section{Modules}
Modules are Python programs that contain functions that we often reuse (we also call them libraries or libraries). These are “toolboxes” that will be very useful to you.\newline

\subsection{Import}
\label{subsec:Import}
\subsubsection{Import module from another repertory}
Sometimes you need to import a module that isn't into you actual repertory, rather than copy this module into you repertory you can import it by another way. \\

Tree exemple :
Main/
	main.py
	Test/
		tes.py

\begin{lstlisting}[language=Python]
import sys
sys.path.append("./Test")
import test # .py file (module)
\end{lstlisting}

\subsubsection{Import module as}
Sometimes you import modules but it's very long to write methods from a long module name. Plus, it's not good to use the \textit{import * from module} because it cans creates conflicts between function and class that have the same name in two different modules. Rather than writing module.class you can use the key word \textbf{as}.
\label{As}

\subsection{Time}
\subsubsection{Time}
\subsubsection{Datetime}
\subsection{Password}
\subsubsection{Getpass}
\subsubsection{Hashlib}
\subsection{Os}

\subsection{Sys}
\subsubsection{Memory used by a variable}
Control you memory while programming. 
\begin{lstlisting}[language=Python]
import sys
var = [1]*100
print(f"memoire utilisee par var : {sys.getsizeof(var)} bytes") #memoire utilisee par var : 856 bytes
\end{lstlisting}

\subsection{Math}
\subsection{Random}
\subsection{Argparse}
\subsection{Unittest}
\subsection{Re}
\subsection{Cx\_freeze / setup}
\subsection{Reseau}
\subsubsection{Socket}
\subsubsection{Select}
\subsection{Threads}
\subsection{Inspect}
If you're interested in knowing how a class is built you can use the inspect module. It prints out the full code from the class.
\begin{lstlisting}[language=Python]
import inspect
from queue import Queue
print(f"Code source : \n{inspect.get_source(Queue)}")
\end{lstlisting}

\newpage
\section{Decorators}
A decorator is a design pattern in Python that allows a user to add new functionality to an existing object without modifying its structure. Decorators are usually called before the definition of a function you want to decorate. \textbf{@my\_decorator}\newline
\subsection{Function}
A decorator can be applied to wrap a function and apply a process on it each time it's called.
\begin{lstlisting}[language=Python]
import time
def timer(func):
	def wrapper(*args, **kwargs):
		start = time.time()
		rv = func(*args, **kwargs)
		total = time.time() - start
		print("Time to execute : "+ str(total))
		return rv
	return wrapper

@timer #we can chain decorators
def exemple(x):
	return "je suis "+str(x)
	
exemple(5)
#Time to execute : 2.384185791015625e-06
#je suis 5
\end{lstlisting}

\subsection{Class}
A decorator can also be used to wrap a class.

\subsection{Multiple decorators}
You can chain decorators to apply multiple processes on the same function / class.

\newpage
\section{Generators}
\label{sec:Generators}
Generators are simple and powerful tools for creating iterators. They are written like regular functions but use the yield statement when they want to return data. Each time it is called by next (), the generator resumes execution where it left off (keeping all of its execution context).\newline
\subsection{Iterators}
\subsection{Yield}

\newpage
\section{Knowledge in bulk}
\subsection{Break / Continue / Pass / Return}
\label{subsec:BCPR}
\begin{lstlisting}[language=Python]
#pass is similar to an empty instruction.
if True:
    pass

#return is used un function to give a response to a call.
def func():
	"""this is a docstring, use help(func) to know more about this function."""
	return "j'ai retourne une phrase"

#break and continue are used in loop to modify the way the loop is looping.
nb = 3
while nb != 20:
	if nb>6:
		break #End the loop
	nb+=1
	if nb==5:
		continue #Go to the top of the loop without executing the rest.	
	print(nb)

#4
#6
#7
\end{lstlisting}

\subsection{Context manager}
\label{subsec:ContextManager}

\subsection{Global variables}
\label{subsec:Global}
Variables that are created outside of a function (as in all of the examples above) are known as global variables.\\
Global variables can be used by everyone, both inside of functions and outside.
\begin{lstlisting}[language=Python]
a = b = 5
#a = 5 et b = 5
def modify(b):
	global a #modify the variable outside the function
	a*=2
	b*=2

modify(b)
#a = 10 et b = 5
\end{lstlisting}

\subsection{Exception}
In python you can handle exceptions that may occur when your program is running. \textit{We can always use \textbf{except Exception as e} but it's better to be the most precise possible.}
\subsubsection{Try / Except / Else / Finally}
This is the basic stucture to handle exception :
\label{TEEF}
\begin{lstlisting}[language=Python]
nom = "Jerome"
try:
	key = int(nom)
except ValueError:
	print("Votre nom ne peut pas etre convertit en integer")
else: #done if try is ok
	print("Key as been validated")
finally: #always done
	print("end ...")
	
#Votre nom ne peut pas etre convertit en integer
#end ...
\end{lstlisting}

We can also create our own exception by creating a new class.
\begin{lstlisting}[language=Python]
#Creer notre propre exception : La methode __str__ de la classe est ce qui est appele pour afficher le message d'erreur
class MonException(Exception):
    """Exception levee dans un certain contexte qui reste a definir"""
    def __init__(self, message):
        """On se contente de stocker le message d'erreur"""
        self.message = message
    def __str__(self):
        """On renvoie le message"""
        return self.message
\end{lstlisting}

\subsubsection{Raise}
\label{Raise}
If you want to personnalize the way a process handle exceptions you can use the \textbf{raise} keyword.


\subsubsection{If / Elif / Else}
\label{IEE}
You can write if / else conditions in an easy way in python :
\begin{lstlisting}[language=Python]
condition = False

#Bad way :

if condition:
	x=1
else:
	x=0
	#0

#Good way :
x = 1 if condition else 0
#0
\end{lstlisting}

\subsection{Files}

\subsection{// and **}
There are two basics operations in python very useful instead of using a math method.
\begin{lstlisting}[language=Python]
print(f"10 / 3 = {10 / 3}") # 10 / 3 = 3.3333333333333335
print(f"10 // 3 = {10 // 3}") # 10 // 3 = 3

print(f"3**3 = {3**3}"}) # 3**3 = 9
\end{lstlisting}

\subsection{Assertion}
\label{subsec:Assertion}
You can verify a condition before parsing an information by checking the instance type.
\begin{lstlisting}[language=Python]
b = 3
assert isinstance(b, int)
\end{lstlisting}

\subsection{Cast}
Sometimes you need to convert a variable from a certain type to another. Cast is a good way to do it but you've to know that you can lose information by doing it.

\subsection{For}
For loop are the most powerful way to work on containers.
\subsubsection{Enumerate}
Enumerate is a very good way to work with for loop. You can access simultaneously to the element and it's position in a tuple.
\begin{lstlisting}[language=Python]
liste = ['a', 'b', 'c']
for index, elem in enumerate(liste):
	print(f"{elem} at position {index}")
# a at position 0
# b at position 1
# c at position 2
\end{lstlisting}

\subsubsection{Zip}
Zip is another loop method. You can combine multiple lists with the same length in the loop. 
\begin{lstlisting}[language=Python]
names = ["Peter Parker", "Clark Kent", "Wade Wilson", "Bruce Wayne"]
heroes = ["Spiderman", "Superman", "Deadpool", "Batman"]
universes = ["Marvel", "DC", "Marvel", "DC"]

for name, hero, universe in zip(names, heroes, universes):
	print(f"{name} is actually {hero} from {universe}")
	
# Peter Parker is actually Spiderman from Marvel
# Clark Kent is actually Superman from DC
# Wade Wilson is actually Deadpool from Marvel
# Bruce Wayne is actually Batman from DC
\end{lstlisting}

\end{document}
